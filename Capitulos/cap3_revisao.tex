\chapter{Revisão da Literatura} 
\label{cap3_revisao} 

\section{Tecnologias Digitais na Educação}

A adoção das tecnologias digitais e das plataformas de ensino no contexto educacional tem-se consolidado como uma tendência irreversível, trazendo novos desafios e oportunidades para o Ensino Superior. A incorporação dessas ferramentas não apenas enriquece o processo de ensino e aprendizagem, mas também amplia as possibilidades de interação, colaboração e acesso a recursos educacionais diversificados. A implementação dessas tecnologias é especialmente relevante, dado o contexto socioeconômico e as especificidades regionais que influenciam diretamente o cotidiano acadêmico \cite{Ferreira_Lopes_Oliveira_Carvalho_Souza_Santos_Silva_Veloso_2024}.

Segundo \citeonline{Fernandes2020}, as plataformas digitais desempenham um papel fundamental na organização e otimização dos processos educacionais, promovendo maior transparência e eficiência no gerenciamento de dados acadêmicos. Essa transformação é especialmente relevante em áreas como o planejamento acadêmico e a distribuição de recursos, impactando diretamente a organização das atividades diárias e o suporte à tomada de decisões no ambiente educacional.

A implementação de sistemas digitais em instituições acadêmicas enfrenta desafios
significativos, como a resistência à mudança por servidores e alunos. A falta de capacitação e questões de segurança de dados também são críticas, além do custo de manutenção e integração com sistemas existentes \cite{senger2022gestao}. 

No IFNMG - Campus Salinas, a proposta de integrar uma plataforma web à planilha atual busca minimizar esses desafios, combinando inovação com familiaridade. \citeonline{garcia2024plataforma} destaca que essa estratégia reduz resistências, facilita a centralização de informações de forma segura, acessível e agradável de navegar.

\section{Sistemas Web}

A tecnologia é indispensável para modernizar e otimizar a gestão escolar, transformando processos e promovendo uma administração mais eficiente \cite{Urania2024}. No contexto do IFNMG - Campus Salinas, a adoção de um sistema web para a organização e exibição de horários acadêmicos visa superar as limitações apresentadas pelo uso de planilhas tradicionais, promovendo maior acessibilidade, organização e segurança.

Um sistema web é uma aplicação que pode variar de páginas simples a sites complexos, sendo acessada por meio de navegadores em diferentes dispositivos conectados à internet \cite{pressman2016engenharia}. Essa característica é especialmente útil em instituições de ensino com uma comunidade acadêmica distribuída, facilitando o acesso simultâneo por alunos, professores e servidores. Segundo \citeonline{silveira2015uef}, a demanda por aplicações web visa atender às diversas necessidades das pessoas e organizações, como apoiar a realização de atividades laborais, comerciais, educacionais e de comunicação, entre outras.

Independentemente do processo adotado, o desenvolvimento de uma aplicação web envolve quatro atividades fundamentais. A primeira fase é a especificação de software, onde as funcionalidades e as restrições do sistema devem ser definidas. Em seguida, passa-se para a fase de projeto e implementação de software, onde o sistema deve ser produzido para atender às especificações. Após a implementação, inicia-se a fase de validação de software, onde o sistema deve ser validado para garantir que atenda às demandas do cliente. Por fim, a fase de evolução de software, onde o sistema deve evoluir para atender às necessidades de mudança dos clientes \cite{sommerville2011engenharia}.

\subsection{Levantamento de Requisitos}

O levantamento de requisitos é uma etapa essencial no processo de desenvolvimento de software, pois define as funcionalidades e as restrições do sistema a ser construído \cite{Balieiro_Pinto_2025}.

De acordo com \citeonline{sommerville2011engenharia}, nessa atividade, os engenheiros de software trabalham com clientes e usuários finais do sistema para obter informações sobre o domínio da aplicação, os serviços que o sistema deve oferecer, o desempenho do sistema, restrições de hardware e assim por diante. Os requisitos podem ser classificados como:

\begin{itemize}
    \item \textbf{Funcionais}: Definem os serviços que o sistema deve fornecer, de como deve reagir a entradas específicas e de como deve se comportar em determinadas situações. Em alguns casos, também podem explicitar o que o sistema não deve fazer.
    \item \textbf{Não Funcionais}: Impõem restrições aos serviços ou funções oferecidos pelo sistema. Incluem restrições de timing, restrições no processo de desenvolvimento e restrições impostas pelas normas. Ao contrário dos requisitos funcionais, muitas vezes, aplicam-se ao sistema como um todo.
\end{itemize}

Neste trabalho, será adotada a técnica de entrevistas, que consiste em conversas estruturadas com os responsáveis pelo setor de ensino. Conforme \citeonline{kendall2011systems}, em todo o desenvolvimento de software, um aspecto fundamental é a captura dos requisitos dos usuários. Uma entrevista de levantamento de informações é uma conversa direcionada com um propósito específico, que utiliza um formato ``pergunta/resposta''. Durante esse processo, busca-se obter as opiniões do entrevistado, conhecer os sentimentos do entrevistado sobre o estado corrente do sistema, obter metas organizacionais e pessoais, além de levantar procedimentos informais para interação com tecnologias da informação.

Em uma entrevista, o engenheiro de software está, provavelmente, estabelecendo um relacionamento com uma pessoa desconhecida. Para que esse processo seja eficaz, deve seguir três etapas principais:

\begin{itemize}
    \item \textbf{Planejamento}: Definir os objetivos da entrevista, estudar materiais sobre os entrevistados e a organização, preparar perguntas e estruturar a entrevista para garantir que todas as informações essenciais sejam abordadas.
    \item \textbf{Condução}: Realizar a entrevista de forma organizada, registrando as respostas por meio de anotações ou gravações, garantindo que o entrevistado se sinta à vontade e forneça informações relevantes.
    \item \textbf{Elaboração do Relatório}: Documentar os principais pontos discutidos, destacando informações críticas e identificando aspectos que precisam ser aprofundados em entrevistas futuras.
\end{itemize}

\subsection{Front-end}

O \textit{front-end} é a interface visual de um site com a qual os usuários interagem diretamente. Engloba todos os elementos visuais, como layouts, botões, imagens, além da lógica de interação do usuário. Essa parte recebe os dados fornecidos pelo \textit{back-end} e os apresenta de maneira intuitiva e compreensível. Ademais, coleta e valida as entradas do usuário antes de enviá-las de volta ao \textit{back-end} \cite{garcia2024plataforma}.

\subsubsection{Projeto de Interface e Protótipos}

A interação com sistemas informatizados se tornou indispensável. Esses sistemas são utilizados como apoio fundamental a muitas atividades diárias, das mais simples às mais complexas. O sucesso desses sistemas é determinado pela qualidade do apoio oferecido aos seus usuários, sendo fortemente influenciado pela facilidade de uso \cite{miletto2014desenvolvimento}.

Garantir a qualidade na construção das interfaces é fundamental para que a interação proporcione confiança e satisfação ao usuário. Confiança, no sentido de que o sistema seja seguro e cumpra sua função com eficácia. Satisfação, por oferecer uma navegação agradável, eficiente e que permita ao usuário atingir seus objetivos com rapidez e menor esforço. Para isso, a concepção de interfaces inclui as seguintes etapas \cite{miletto2014desenvolvimento}:
\begin{itemize}
    \item \textbf{Engenharia de requisitos}: A estrutura do site e o contexto de utilização são identificados.
    \item \textbf{Especificação}: Modelos da interface são construídos a partir dos requisitos obtidos durante a fase de análise.
    \item \textbf{Design}: Modelos são construídos para que se possa refletir sobre o site e suas propriedades sem ter que implementá-lo.
    \item \textbf{Implementação}: Criação de páginas HTML e de objetos de som/imagem necessários à aplicação.
    \item \textbf{Utilização e avaliação}: Usabilidade da interface e a sua coerência em relação aos requisitos iniciais são avaliadas.
    \item \textbf{Manutenção}: Envolve um ciclo de maior duração, que abrange a coleta de novos requisitos e o planejamento das modificações identificadas durante a etapa de avaliação.
\end{itemize}

\subsubsection{Ferramenta para Design e Prototipação}

O \textit{Figma}\footnote{https://www.figma.com} é uma ferramenta de design baseada na nuvem que permite criar interfaces de usuário e simular sua interação de forma eficiente \cite{Figma2025}. Entre suas principais funcionalidades, destacam-se:

\begin{itemize}
    \item \textbf{Colaboração em tempo real}: Permite que vários membros da equipe acessem e editem o mesmo projeto simultaneamente, facilitando a troca de ideias e o alinhamento durante o desenvolvimento.
    \item \textbf{Protótipos interativos}: É possível simular a navegação e o funcionamento da interface antes de implementar o código, permitindo validar conceitos e fluxos de trabalho.
    \item \textbf{Facilidade de ajustes}: A interface intuitiva do \textit{Figma} possibilita realizar alterações rápidas no design, garantindo que os protótipos estejam alinhados às necessidades dos usuários e aos objetivos do projeto.
\end{itemize}

\subsubsection{Ferramentas e Tecnologias para Desenvolvimento Front-end}

\begin{itemize}
    \item \textbf{TypeScript}: É uma linguagem de programação que se baseia no \textit{JavaScript}, adicionando tipagem estática e recursos avançados de orientação a objetos. Foi projetada para criar aplicações mais organizadas e escaláveis, além de oferecer uma experiência de desenvolvimento mais segura, permitindo que erros sejam detectados antes mesmo da execução do código \cite{microsoft2025}.
    \item \textbf{Bibliotecas e frameworks do lado do cliente}: A definição de \textit{framework} pode variar de acordo com a pesquisa, existindo assim, várias definições na literatura. Um \textit{framework} é um conjunto de classes que colaboram entre si de modo a prover um reúso abrangente, de grandes blocos de comportamento. Existem muitas bibliotecas de \textit{frameworks} \textit{JavaScript} disponíveis para os desenvolvedores de software trabalharem, cada um é único em sua própria maneira, enquanto muitos farão algumas das mesmas coisas, mas muitas vezes de forma diferente. De modo geral, tendem a facilitar os desenvolvedores na construção de aplicações \cite{silva2022comparaccao}.
    \begin{itemize}
        \item \textbf{React.js}: É uma biblioteca para construção de interfaces de usuário \cite{react2025}. As duas principais vantagens do \textit{React} são permitir criar componentes reutilizáveis de interface, a fim de reduzir linhas de código e a alteração de elementos ou dados exibidos sem a necessidade de recarregar a página \cite{LorenaUFOP2021}.
        
        A componentização do \textit{React} serve para dividir a página em vários componentes com funções específicas. É considerado um componente quando o mesmo pode ser isolado sem interferir no restante do código facilitando a manutenção. Com base nisso, é uma biblioteca para facilitar a construção de interfaces e de single-page applications \cite{LorenaUFOP2021}.
        
        \item \textbf{Next.js}: É um \textit{framework} \textit{React} de código aberto hospedado no \textit{GitHub} sob a licença MIT, com foco em produção e eficiência, criado e mantido pela equipe da \textit{Vercel} \cite{nextjs2025}. O \textit{Next.js}\footnote{https://nextjs.org} tem se destacado por sua capacidade de renderização no lado servidor (SSR), melhorando significativamente o tempo de carregamento das páginas e a otimização para motores de busca \cite{RenanNextjs2024}. Ademais, oferece recursos como geração de site estático (SSG), roteamento baseado em sistema de arquivos, suporte a CSS e SASS incorporados, além de suportar \textit{TypeScript}. Sua flexibilidade permite que os desenvolvedores escolham entre SSR, SSG ou uma combinação de ambos, dependendo das necessidades do projeto \cite{RenanNextjs2024}.
    \end{itemize}
    \item \textbf{Tailwind CSS}: É um framework de CSS utilitário que permite aos desenvolvedores aplicar estilos diretamente nos componentes HTML, facilitando a criação de designs responsivos e personalizados. O \textit{Tailwind CSS}\footnote{https://tailwindcss.com} se destaca por fornecer classes utilitárias que podem ser combinadas para construir qualquer design. Essa abordagem aumenta a produtividade e a consistência dos projetos de desenvolvimento web \cite{bosco2024crefide}.
\end{itemize}

\subsection{Back-end}

Enquanto o \textit{front-end} é a parte visível de um site, o \textit{back-end} é a parte responsável por gerenciar as solicitações do cliente e fornecer os dados necessários para a interface. Essa parte é crucial para garantir que as informações sejam armazenadas com segurança, que as operações sejam executadas de forma eficiente e que as respostas aos usuários sejam rápidas e precisas. Além disso, cuida da lógica de negócios, acesso ao banco de dados e autenticação de usuários \cite{garcia2024plataforma}.

\subsubsection{Planilhas como Banco de Dados}

O uso de planilhas em nuvem como forma de armazenamento de dados para aplicações web pode parecer, à primeira vista, uma escolha incomum ou até controversa, considerando que as abordagens tradicionais costumam recorrer a sistemas gerenciadores de banco de dados. No entanto, essa alternativa tem sido explorada em diversos contextos, como nas áreas de mídia, jornalismo e até mesmo no desenvolvimento de jogos \cite{schwertnercharao:hal-02119998}.

Adotar essa estratégia pode ser vantajoso devido à interface web simples, eficaz e amplamente conhecida para o gerenciamento de dados. As planilhas suportam operações de inserção, atualização e exclusão, que precisariam ser implementadas para a gestão dos registros. Além disso, por serem projetadas para colaboração, essas ferramentas oferecem recursos fáceis de gerenciamento de acesso, permitindo conceder permissões de leitura e escrita \cite{schwertnercharao:hal-02119998}.

Apesar dessas vantagens, essa abordagem apresenta limitações importantes. Há um controle limitado sobre a integridade e o formato dos dados, dificultando a padronização das informações. Outro ponto crítico é a propensão a erros humanos, como sobrescrever células por engano ou registrar dados em estruturas inconsistentes, comprometendo a confiabilidade do sistema \cite{wisconsin2020}.

Considerando que muitas soluções para gerenciamento de banco de dados exigem hospedagem paga, isso pode se tornar inacessível para projetos estudantis ou pessoais. Nesse contexto, as planilhas surgem como uma alternativa funcional para o armazenamento e a manipulação de dados \cite{ufsm2024}.

\subsubsection{Ferramentas e Tecnologias para Desenvolvimento Back-End}

\begin{itemize}
    \item \textbf{Java}: É uma linguagem de programação orientada a objetos e plataforma de desenvolvimento de software. Tem como uma de suas principais características o fato de ser multiplataforma, ou seja, um mesmo código escrito em \textit{Java} pode ser executado em diferentes plataformas e sistemas operacionais, bastando que a máquina possua a \textit{JVM} instalada, sendo apenas uma parte que está envolvida na execução de um aplicativo \cite{Java2025}.
    \item \textbf{Spring}: É um \textit{framework} que simplifica o desenvolvimento de software em \textit{Java}, destacando-se por ser altamente configurável e por suportar variados tipos de arquiteturas, já que é dividido em módulos que podem ser usados de acordo com as necessidades da aplicação \cite{Spring2025}.
    \item \textbf{Spring Boot}: É um dos módulos construídos sobre o \textit{Spring Framework}. Ao utilizá-lo, a configuração de aplicações \textit{Spring}\footnote{https://spring.io/quickstart} é facilitada para o desenvolvedor, pois realiza configurações de forma automática. Dessa forma, o desenvolvedor pode direcionar sua atenção mais para as regras de negócio e menos para configurações de projeto, ajudando assim a acelerar o processo de desenvolvimento \cite{SpringBoot2025}.
    \item \textbf{Google Sheets}: É um aplicativo de planilhas online que permite a criação, formatação e edição de arquivos de forma colaborativa. Embora frequentemente associado à criação de planilhas, o \textit{Google Sheets}\footnote{https://docs.google.com/spreadsheets} também pode ser uma alternativa funcional para o armazenamento e a manipulação de dados \cite{ufsm2024}.
    \item \textbf{Google Sheets API}: É uma interface \textit{RESTful} que permite ler e modificar dados de uma planilha do \textit{Google Sheets} sendo disponibilizada pelo Google Cloud Platform \cite{apisheets2025}. Por meio dessa \textit{API}, os desenvolvedores podem automatizar tarefas e integrar planilhas a sistemas externos. Entre suas principais funcionalidades estão:

    \begin{itemize}
        \item \textbf{Criar planilhas}: Permite a criação automatizada de novas planilhas diretamente pelo sistema.
        \item \textbf{Ler e gravar valores de células de planilhas}: Possibilita a extração e inserção de dados em células específicas, permitindo atualização dinâmica das informações.
        \item \textbf{Atualizar a formatação da planilha}: Modifica estilos, cores, tamanhos de fonte e outras propriedades visuais da planilha para melhor organização e apresentação dos dados.
        \item \textbf{Gerenciar páginas conectadas}: Adiciona, renomeia, reorganiza e exclui guias dentro da planilha, facilitando a estruturação dos dados conforme as necessidades do sistema.
    \end{itemize}

    A integração da ferramenta com planilhas oferece uma maneira prática de gerenciar dados, automatizar processos e colaborar com outros usuários. Com isso, o \textit{Google Sheets} se torna uma solução acessível que pode servir como uma alternativa viável a banco de dados para muitos tipos de projetos. No entanto, é fundamental compreender suas limitações e adotar boas práticas para garantir que atenda às necessidades do projeto de forma eficiente e segura \cite{ufsm2024}.
    
    \item \textbf{Docker}: É uma plataforma aberta para desenvolver, enviar e executar aplicações em contêineres. Os contêineres são unidades leves e portáteis que incluem tudo o que uma aplicação precisa para funcionar, como bibliotecas, dependências e configurações. Ao aproveitar suas metodologias, é possível reduzir significativamente o atraso entre a escrita e a execução do código na produção \cite{docker2025}.
\end{itemize}

\subsubsection{Padrão de Arquitetura MVC}

O MVC é um padrão de arquitetura de software amplamente adotado, sendo a base do gerenciamento de interação em muitos sistemas baseados na web. Seu principal objetivo é promover uma separação entre a apresentação e a interação dos dados do sistema \cite{sommerville2011engenharia}. Essa estrutura organiza o sistema em três componentes lógicos que interagem entre si:

\begin{itemize}
    \item \textbf{Modelo (Model)}: gerencia os dados e as operações associadas a esses dados.
    \item \textbf{Visão (View)}: define e gerencia como os dados são apresentados ao usuário.
    \item \textbf{Controlador (Controller)}: gerencia a interação do usuário e passa essas interações para a Visão e o Modelo.
\end{itemize}

É indicado em projetos que demandam várias maneiras de se visualizar e interagir com dados, ou quando são desconhecidos os futuros requisitos de interação e apresentação de dados \cite{sommerville2011engenharia}.

Entre suas principais vantagens, permite que os dados sejam alterados de forma independente de sua representação, além de apoiar a apresentação dos mesmos dados de maneiras diferentes. Por outro lado, quando o modelo de dados e as interações são simples, pode envolver código adicional e complexidade de código \cite{sommerville2011engenharia}.

\subsubsection{Integração Front-end e Back-end}

A conexão entre \textit{front-end} e \textit{back-end} é um aspecto fundamental no desenvolvimento de software. A integração eficiente entre essas duas partes é crucial para garantir que o projeto funcione corretamente, proporcione uma boa experiência ao usuário, além de contribuir para a qualidade dos produtos e a satisfação dos clientes \cite{da2024guia}. Assim, destacam-se os principais conceitos e práticas envolvidos nesse processo:

\begin{itemize}
    \item \textbf{Requisições HTTP}: O termo HTTP é utilizado para se referir a um protocolo de comunicação entre aplicações web. Esse protocolo define as regras de como devem ser feitas as requisições ao servidor para que a transferência de dados seja bem sucedida entre o cliente e o servidor \cite{portilho2021desenvolvimento}.
    \item \textbf{API REST}: É um conjunto definido de mensagens de requisição e resposta HTTP, geralmente expressado nos formatos XML ou JSON. Muitas vezes se comunica com diversos outros códigos interligando diversas funções em um aplicativo \cite{oliveira2014desenvolvimento}.
    \item \textbf{Serialização de dados}: É um processo essencial que envolve a conversão de estruturas de dados complexas, como objetos, em um formato que possa ser facilmente armazenado e transmitido. Esse processo é crucial em cenários onde é necessário preservar a integridade dos dados ao longo do tempo, transmitir informações entre diferentes sistemas ou compartilhar dados com outras partes \cite{castilhos2024analise}.
    \item \textbf{Autenticação e autorização}: Em muitas aplicações web que utilizam segurança, geralmente é necessário fornecer mecanismos para que o cliente possa se identificar. A partir desta identificação, o servidor checa se as credenciais fornecidas são válidas e quais recursos o cliente está autorizado a acessar. Estes procedimentos recebem o nome de autenticação, que ocorre quando o cliente fornece as credenciais válidas, e autorização, em que o servidor verifica os direitos de acesso do cliente \cite{oliveira2014desenvolvimento}.
    \item \textbf{JSON}: É um formato leve de troca de dados que é fácil para os humanos lerem e escreverem, e fácil para as máquinas analisarem e gerarem. É baseado em formato de texto e completamente independente de linguagem, pois usa convenções que são familiares às linguagens C e muitas outras. Essas propriedades tornam este um formato ideal para transmissão de dados em aplicações web \cite{json2025}.
\end{itemize}

\subsubsection{Avaliação de Software}

A avaliação de um software tenta compreender o estado atual do processo de desenvolvimento com o intuito de aperfeiçoá-lo. Essa etapa é fundamental para verificar se o sistema atende às necessidades reais dos usuários e identificar pontos que podem ser aprimorados. Nesse sentido, informações de uma ampla variedade de fontes, como entrevistas com usuários e dados gerenciais, podem ser utilizadas para concretizar esse processo de validação \cite{pressman2016engenharia}.