\chapter{Introdução} 
\label{cap1_introducao} 

Todos os departamentos de ensino enfrentam, no início de cada semestre acadêmico, a necessidade de criar horários escolares. A elaboração de um horário acadêmico é uma tarefa complexa, que envolve a gestão de restrições entre alunos, professores e salas em determinados períodos letivos, e deve ser apoiada por sistemas de informação avançados. A combinação de sistemas capazes de gerenciar informações acadêmicas e técnicas de pesquisa e otimização eficientes pode possibilitar a geração semiautomática de horários acadêmicos, com um desempenho que, na maioria das vezes, seria difícil de alcançar com um processo tradicional, essencialmente manual \cite{passos2016}.

O IFNMG - Campus Salinas é uma instituição pública de ensino que se destaca em oferecer educação profissional e tecnológica em diversos níveis, do ensino médio ao superior \cite{ifnmgsalinas2014}. Nos ambientes acadêmicos, a gestão de horários desempenha um papel central na organização das atividades, sendo essencial para o bom funcionamento de instituições de ensino. A eficiência no acesso a essas informações, tanto para alunos quanto para servidores, impacta diretamente a rotina acadêmica e administrativa \cite{Urania2024}. 

Entretanto, em muitas instituições, como o IFNMG - Campus Salinas, a organização dos horários é realizada por meio de planilhas. Embora amplamente utilizadas, apresentam limitações que dificultam a consulta, a manutenção e a proteção das informações, criando um cenário desafiador para a administração acadêmica.

A transformação digital, cada vez mais presente no setor educacional, demonstra que informatizar processos administrativos deixou de ser uma conveniência para se tornar uma estratégia essencial. Em ambientes acadêmicos, o uso de plataformas digitais não apenas melhora a eficiência, como também otimiza a gestão de informações críticas \cite{Educacional2023}. Assim, segundo o professor Frederico Ventura Batista, Coordenador de Ensino Superior/Diretor de Ensino Substituto do Campus Salinas:

\begin{citacao}
Além da confecção dos horários, que se mostra sempre desafiadora, a apresentação da planilha à comunidade acadêmica e sua manutenção têm se tornado cada vez mais difíceis. A necessidade de permitir que usuários visualizem os horários de forma clara é prejudicada pelas restrições da planilha em uso. Com design pouco intuitivo, falta de responsividade em dispositivos móveis e dificuldade na busca por horários de um curso específico. Tais limitações deixam claro que a modernização do sistema de exibição de horários é fundamental para garantir maior eficiência e acessibilidade.
\end{citacao}

Antes da proposta deste trabalho, a organização dos horários acadêmicos no IFNMG - Campus Salinas era realizada por meio de uma planilha disponibilizada no Google Sheets. O acesso era compartilhado com todos os usuários da instituição por meio de um link aberto, permitindo a visualização e também a edição irrestrita dos dados. Embora a estrutura da planilha não fosse inadequada, apresentava limitações significativas de usabilidade, especialmente na visualização em dispositivos móveis e na busca por informações específicas, como horários por curso. Além disso, a possibilidade de edição por qualquer usuário comprometia a integridade dos dados, gerando riscos de alterações não autorizadas e aumentando a necessidade de manutenção frequente. Não havia uma plataforma dedicada para gerenciar esses horários, o que tornava o processo vulnerável e pouco eficiente.

Tendo isso em vista, o presente trabalho propõe uma solução em formato de plataforma web, com o potencial de otimizar a gestão dos horários acadêmicos no IFNMG - Campus Salinas. Com uma interface intuitiva e segura, o sistema facilitará a consulta e a visualização das informações, ao mesmo tempo em que impedirá alterações não autorizadas, garantindo a integridade dos dados. Dessa forma, o setor de ensino poderá gerenciar as informações de forma mais eficiente, sem precisar refazer processos ou depender de bancos de dados complexos.

Por fim, a metodologia do trabalho é baseada no desenvolvimento de um sistema que integra tecnologias modernas para exibição dos horários acadêmicos. Durante o processo, foram realizadas etapas como levantamento de requisitos, prototipação da interface e implementação do sistema. A solução foi desenvolvida utilizando \textit{Next.js} para o \textit{front-end}, com integração ao \textit{Tailwind CSS}, enquanto o \textit{back-end} foi construído com \textit{Spring Boot}, garantindo a conexão com o \textit{Google Sheets}, que funciona como banco de dados. 

A escolha das tecnologias levou em conta a praticidade, a familiaridade do desenvolvedor e a eficiência na integração entre as ferramentas. No front-end, adotou-se o Next.js, por simplificar a configuração de rotas e melhorar o desempenho de aplicações React, aliado à facilidade de uso em projetos atuais. O Tailwind CSS foi integrado devido à compatibilidade com o Next.js e à agilidade que oferece na construção de interfaces responsivas. A utilização do TypeScript proporcionou maior segurança e organização no código, por meio de sua tipagem estática, enquanto o React.js foi escolhido pela experiência prévia do desenvolvedor e pela capacidade de criar interfaces dinâmicas.

No back-end, a linguagem Java foi utilizada em conjunto com o Spring Framework, que facilita a criação de aplicações robustas, e o Spring Boot, que simplifica a configuração e acelera o desenvolvimento. Para o armazenamento das informações, optou-se pelo Google Sheets, funcionando como um banco de dados. A integração com o sistema foi realizada por meio da Google Sheets API, permitindo a leitura dos dados da planilha. Além disso, para garantir a portabilidade e o isolamento do ambiente de execução, foi empregada a conteinerização com o Docker, facilitando o deploy em diferentes ambientes.

Em etapas futuras, será realizada a avaliação da plataforma com os responsáveis pela gestão dos horários acadêmicos, visando coletar \textit{feedbacks} para ajustes e melhorias. Ademais, será elaborada uma documentação técnica para definir diretrizes sobre elementos cruciais do sistema, assegurando sua continuidade e manutenção.

O resto do documento está organizado da seguinte forma: O capítulo \ref{cap2_objetivos} apresenta os elementos principais, como referências, figuras, tabelas, equações, algoritmos e outros elementos explicativos. O capítulo \ref{cap3_revisao} apresenta um texto fictício aleatório. O capítulo \ref{cap4_metodologia} apresenta um texto fictício aleatório. O capítulo \ref{cap5_resultados} apresenta alguns comentários e observações finais. Por último, veja um exemplo do Apêndice apendiceA ou do Anexo anexoA.


