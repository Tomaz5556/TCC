\chapter{Introdução} 
\label{cap1_introducao} 

Todos os departamentos de ensino enfrentam, no início de cada semestre acadêmico, a necessidade de criar horários escolares. A elaboração de um horário acadêmico é uma tarefa complexa, que envolve a gestão de restrições entre alunos, professores e salas em determinados períodos letivos, e deve ser apoiada por sistemas de informação avançados. A combinação de sistemas capazes de gerenciar informações acadêmicas e técnicas de pesquisa e otimização eficientes pode possibilitar a geração semiautomática de horários acadêmicos, com um desempenho que, na maioria das vezes, seria difícil de alcançar com um processo tradicional, essencialmente manual \cite{passos2016}.

O IFNMG - Campus Salinas é uma instituição pública de ensino que se destaca em oferecer educação profissional e tecnológica em diversos níveis, do ensino médio ao superior \cite{ifnmgsalinas2014}. Nos ambientes acadêmicos, a gestão de horários desempenha um papel central na organização das atividades, sendo essencial para o bom funcionamento de instituições de ensino. A eficiência no acesso a essas informações, tanto para alunos quanto para servidores, impacta diretamente a rotina acadêmica e administrativa \cite{Urania2024}. Entretanto, em muitas instituições, como o IFNMG - Campus Salinas, a organização dos horários é realizada por meio de planilhas. Embora amplamente utilizadas, apresentam limitações que dificultam a consulta, a manutenção e a proteção das informações, criando um cenário desafiador para a administração acadêmica.

A transformação digital, cada vez mais presente no setor educacional, demonstra que informatizar processos administrativos deixou de ser uma conveniência para se tornar uma estratégia essencial. Em ambientes acadêmicos, o uso de plataformas digitais não apenas melhora a eficiência, como também otimiza a gestão de informações críticas \cite{Educacional2023}. De acordo com o Coordenador de Ensino Superior/Diretor de Ensino Substituto do Campus Salinas:

\begin{citacao}
Além da confecção dos horários, que se mostra sempre desafiadora, a apresentação da planilha à comunidade acadêmica e sua manutenção têm se tornado cada vez mais difíceis. A necessidade de permitir que usuários visualizem os horários de forma clara é prejudicada pelas restrições da planilha em uso. Com design pouco intuitivo, falta de responsividade em dispositivos móveis e dificuldade na busca por horários de um curso específico. Tais limitações deixam claro que a modernização do sistema de exibição de horários é fundamental para garantir maior eficiência e acessibilidade.
\end{citacao}

Antes da proposta deste trabalho, a organização dos horários acadêmicos no IFNMG - Campus Salinas era realizada por meio de uma planilha no \textit{Google Sheets}. O acesso era compartilhado com todos os usuários da instituição por meio de um link aberto, configurado originalmente apenas para visualização. No entanto, foram relatados problemas recorrentes de segurança, nos quais algumas pessoas conseguiam realizar edições indevidas, mesmo sem autorização para isso. Tais falhas comprometiam a integridade dos dados e exigiam intervenções constantes da equipe responsável.

Embora a estrutura da planilha atendesse minimamente à função proposta, apresentava limitações significativas de usabilidade, especialmente na visualização em dispositivos móveis e na busca por informações específicas, como horários por curso. A ausência de uma plataforma dedicada para gerenciar essas informações tornava o processo vulnerável e pouco eficiente.

Tendo isso em vista, o presente trabalho propõe uma solução em formato de plataforma web, com o potencial de otimizar a gestão dos horários acadêmicos no IFNMG - Campus Salinas. Com uma interface intuitiva e mais segura, o sistema facilita a consulta e a visualização das informações, ao mesmo tempo em que impede alterações não autorizadas, garantindo a integridade dos dados. Dessa forma, o setor de ensino poderá gerenciar as informações de forma mais eficiente, sem precisar refazer processos ou depender de bancos de dados complexos.

Por fim, a metodologia do trabalho é baseada no desenvolvimento de um sistema que integra tecnologias modernas para exibição dos horários acadêmicos. Durante o processo, foram realizadas etapas como levantamento de requisitos, prototipação da interface e implementação do sistema. A solução foi desenvolvida utilizando \textit{Next.js} para o \textit{front-end}, com integração ao \textit{Tailwind CSS}, enquanto o \textit{back-end} foi construído com \textit{Spring Boot}, garantindo a conexão com o \textit{Google Sheets}, que funciona como banco de dados.

A primeira versão do sistema foi implantada no segundo semestre de 2024 e, desde então, tem sido amplamente utilizado pela comunidade acadêmica do campus. A adoção da plataforma tem demonstrado ganhos significativos em acessibilidade, organização e segurança dos dados, tornando o processo de consulta dos horários mais eficiente e confiável.

Além disso, foi realizada a avaliação da plataforma com o responsável pela gestão dos horários acadêmicos, visando coletar feedbacks para ajustes e melhorias. Foram implementadas melhorias e atualizações a partir das sugestões e críticas obtidas durante essa avaliação. Ademais, foi elaborada uma documentação técnica da planilha utilizada como banco de dados, assegurando sua manutenção e continuidade.

O presente trabalho está organizado em seis capítulos. No Capítulo \ref{cap2_objetivos}, são definidos o objetivo geral e os objetivos específicos da pesquisa. O Capítulo \ref{cap3_revisao} apresenta a revisão da literatura que embasa o desenvolvimento do trabalho. O Capítulo \ref{cap4_metodologia} descreve a metodologia adotada ao longo do projeto. O Capítulo \ref{cap5_resultados} foca na apresentação e análise dos resultados obtidos. Por fim, o Capítulo \ref{cap6_conclusao} reúne as principais conclusões e reflexões decorrentes do estudo.