\chapter{Conclusão} 
\label{cap6_conclusao} 

O presente trabalho teve como foco avaliar e melhorar uma plataforma web para exibição de horários acadêmicos no IFNMG Campus Salinas. A pesquisa abordou a necessidade de modernização no acesso a essas informações, considerando as limitações de usabilidade, segurança e organização presentes no modelo anterior baseado em planilhas no Google Sheets. A solução proposta manteve as planilhas como base de dados, porém promoveu melhorias significativas na forma como essas informações são acessadas, visualizadas e protegidas.

A relevância do trabalho se destaca por apresentar uma abordagem alternativa de banco de dados para sistemas web. A escolha dessa proposta refletiu uma necessidade real do setor de ensino da instituição e, ao mesmo tempo, representou um desafio técnico e prático que contribuiu significativamente para o desenvolvimento acadêmico e profissional.

Os objetivos foram plenamente alcançados. A plataforma foi desenvolvida com os conhecimentos obtidos durante o curso, avaliada por quem administra os horários acadêmicos e analisada quanto à sua viabilidade técnica. Além disso, foi elaborada uma documentação técnica da planilha utilizada como banco de dados, com diretrizes que orientam a manutenção e a continuidade do sistema, contribuindo diretamente para a melhoria do acesso e da organização dos horários acadêmicos da instituição.

Os resultados obtidos reforçam a efetividade da proposta, com a criação de telas específicas para cursos, professores e salas, além da implementação de uma tela de validação de dados com acesso restrito. A plataforma incorporou recursos de navegação mais claros e objetivos, sendo acompanhada de uma documentação técnica que assegura sua manutenção e continuidade. Tais melhorias contribuíram para aumentar a confiabilidade, acessibilidade e organização das informações, alinhando-se às demandas reais do setor de ensino.

Uma das principais dificuldades encontradas foi a escassez de materiais sobre o uso de planilhas como banco de dados em sistemas web, exigindo maior esforço de pesquisa e testes durante o desenvolvimento.

Este trabalho proporcionou uma experiência de aprendizado rica e desafiadora. Ao explorar uma abordagem alternativa de banco de dados e integrá-la com tecnologias modernas de desenvolvimento web, foi possível aplicar os conhecimentos adquiridos durante a formação de maneira prática, criativa e com impacto direto na realidade da instituição.

Por fim, como sugestão para trabalhos futuros, implementar uma funcionalidade que permita a edição dos horários diretamente pela interface da plataforma. Com essa melhoria, seria possível realizar modificações sem a necessidade de acessar as planilhas, facilitando ajustes pontuais e otimizando o tempo de manutenção do sistema.

Para auxiliar o desenvolvimento de trabalhos futuros, o código-fonte da plataforma apresentada no trabalho está disponível em: \url{https://github.com/Tomaz5556/Horarios-IFNMG-Salinas}.