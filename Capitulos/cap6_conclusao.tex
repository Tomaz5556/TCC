\chapter{Conclusão} 
\label{cap6_conclusao} 

O desenvolvimento e a avaliação desta plataforma web para exibição de horários acadêmicos no IFNMG – Campus Salinas demonstraram que a adoção de tecnologias modernas pode transformar significativamente a forma como essas informações são disponibilizadas e consultadas pela comunidade acadêmica.

Ao longo do projeto, foi possível integrar de forma segura o Google Sheets como base de dados, utilizando o Spring Boot no back-end e o Next.js com Tailwind CSS no front-end, resultando em uma solução robusta, responsiva e de fácil manutenção. A aplicação, além de organizar e centralizar os horários em uma interface intuitiva, aumentou a segurança e eliminou problemas recorrentes de edição indevida na planilha original.

A avaliação realizada com o setor responsável pelos horários acadêmicos evidenciou melhorias importantes na usabilidade, eficiência e acessibilidade, consolidando a proposta como uma alternativa viável e sustentável. As atualizações implementadas após o feedback reforçam o compromisso com a evolução contínua do sistema e com as necessidades reais dos usuários.

Durante o desenvolvimento deste trabalho, pude aprender uma nova abordagem de banco de dados utilizando planilhas, compreendendo na prática como integrá-las de forma eficiente a um sistema web. Esse aprendizado ampliou minhas perspectivas sobre soluções simples e criativas que podem ser aplicadas em projetos futuros.

Como melhoria futura, vislumbra-se a implementação de uma funcionalidade que permita editar os horários diretamente na interface da plataforma, integrando essas alterações ao Google Sheets de forma controlada e segura. Tal recurso traria ainda mais praticidade e reduziria o tempo de manutenção, agregando valor à solução proposta.

Por fim, este trabalho contribuiu para otimizar a rotina de gerenciamento de horários no campus, trazendo ganhos práticos para alunos e servidores, além de apresentar um modelo que pode ser adaptado e replicado em outras instituições de ensino que enfrentam desafios semelhantes.