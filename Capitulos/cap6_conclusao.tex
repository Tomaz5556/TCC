\chapter{Conclusão} 
\label{cap6_conclusao} 

O presente trabalho teve como foco avaliar e melhorar uma plataforma web para exibição de horários acadêmicos no IFNMG Campus Salinas. A pesquisa abordou a necessidade de modernização no acesso a essas informações, considerando as limitações de usabilidade, segurança e organização presentes no modelo anterior baseado em planilhas abertas no Google Sheets. A proposta desenvolvida foi uma solução web simples, eficiente e segura, mantendo o uso das planilhas como base de dados, porém aprimorando significativamente sua forma de exibição e controle.

A relevância desse tema se destaca por apresentar uma abordagem alternativa de banco de dados para sistemas web. A escolha do tema refletiu uma necessidade real do setor de ensino da instituição e, ao mesmo tempo, representou um desafio técnico e prático que contribuiu significativamente para o desenvolvimento acadêmico e profissional.

Os objetivos foram plenamente alcançados. A plataforma foi desenvolvida com base nos conhecimentos adquiridos ao longo do curso, avaliada junto aos responsáveis pelos horários acadêmicos, e sua viabilidade técnica analisada. Além disso, foi elaborada uma documentação completa para orientar a manutenção do sistema, contribuindo diretamente para a melhoria do acesso e da organização dos horários acadêmicos da instituição.

Os resultados obtidos reforçam a efetividade da proposta, com a criação de interfaces específicas para cursos, professores e salas, além da implementação de uma área de validação de dados com acesso restrito. A plataforma passou a contar com recursos de navegação mais claros e objetivos, sendo acompanhada de uma documentação técnica que assegura sua continuidade e manutenção. Tais melhorias contribuíram para aumentar a confiabilidade, acessibilidade e organização das informações, alinhando-se às demandas reais do setor de ensino.

Uma das principais dificuldades encontradas foi a escassez de materiais sobre o uso de planilhas como banco de dados em sistemas web, exigindo maior esforço de pesquisa e testes durante o desenvolvimento.

Este trabalho proporcionou uma experiência de aprendizado rica e desafiadora. Ao explorar uma abordagem alternativa de banco de dados e integrá-la com tecnologias modernas de desenvolvimento web, foi possível aplicar os conhecimentos adquiridos durante a formação de maneira prática, criativa e com impacto direto na realidade da instituição. Essa vivência certamente contribuirá para futuros projetos e soluções na área de desenvolvimento de sistemas.

Por fim, como sugestão para trabalhos futuros, implementar uma funcionalidade que permita a edição dos horários pela interface da plataforma. Com essa melhoria, seria possível realizar alterações sem alterar os dados diretamente nas planilhas.

Por fim, como sugestão para trabalhos futuros, implementar uma funcionalidade que permita a edição dos horários diretamente pela interface da plataforma. Com essa melhoria, seria possível realizar modificações sem a necessidade de acessar as planilhas.